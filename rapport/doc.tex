\documentclass[a4paper, twoside]{article}
\usepackage[utf8]{inputenc}
\usepackage[english,french]{babel}
\title{Traitement du Langage Naturel et Linguistique}

\date{M2 IAAA 2021/2022}

\author{LE BELLEGO Victor, MARTZLOFF Alice, MOUSSOX Vincent}

\usepackage[a4paper]{geometry}
\usepackage{csvsimple}
\usepackage{parskip}
\usepackage{tabulary}
\geometry{hscale=0.80,vscale=0.85,centering}

\usepackage{float}
\usepackage{amssymb}
\usepackage{amsmath}
\usepackage{enumitem}
\usepackage{graphicx}
\usepackage{caption}
\usepackage{subfigure}
\usepackage{wrapfig}
\usepackage{listings}
\usepackage{stmaryrd} 
\usepackage{setspace}
\usepackage{hyperref}
\usepackage{booktabs}
\usepackage{environ}


%%% Minted + options %%%
\usepackage{minted}
\usepackage[x11names,dvipsnames,table]{xcolor}
\usepackage{colortbl}
\usepackage{mdframed}
\definecolor{bgLightGray}{RGB}{240,240,240}
\definecolor{lightgray}{gray}{.95}
\renewcommand{\theFancyVerbLine}{\normalfont {\footnotesize {\arabic{FancyVerbLine}}}}
\let\oldPYGdefault\PYGdefault
\def\PYGdefault#1#2{\hbox{\oldPYGdefault{#1}{#2}}\allowbreak{}}
\surroundwithmdframed[topline = false, leftline = true, rightline = false, bottomline = false,backgroundcolor=bgLightGray,linewidth=0.5pt]{minted}

\newcommand{\inputmintedcustom}[1]{\begingroup \catcode`_=12 \texttt{#1} \begin{mdframed}[topline = false, leftline = true, rightline = false, bottomline = false,backgroundcolor=bgLightGray,linewidth=0.5pt]
\inputminted[linenos=true, tabsize=4, fontsize=\small, xleftmargin=0pt, xrightmargin=5pt, breaklines=true, obeytabs=true, numbersep=5mm,]{python}{#1}
\end{mdframed}}

\newenvironment{mintedcustom}
{%
    \VerbatimEnvironment
    \begin{minted}[linenos=true, tabsize=4, fontsize=\small, xleftmargin=0pt, xrightmargin=5pt, breaklines=true, obeytabs=true, numbersep=5mm]{python}%
}
{%
    \end{minted}%
    }

\RequirePackage{luacolor}
\RequirePackage{pgf}
\definecolor{bgLightGray}{RGB}{235,235,235}
\definecolor{DarkGrey}{rgb}{0.15,0.15,0.15}
\pgfkeys{
 /consoletext/.is family, /consoletext,
 caption/.estore in = \consoletextCaption,
 label/.estore in = \consoletextLabel,
}

\newenvironment{consoletext}[1][]
{	\def\tmp{#1}%
    \pgfkeys{/consoletext,#1}
    \setlength{\OuterFrameSep}{0pt}						% no frame around the text
	\setlength{\FrameSep}{1mm}							% just a bit of colored space around the text
    \colorlet{shadecolor}{DarkGrey}                % background color to display console
	\begin{shaded}\begin{raggedright}\captionsetup{type=consoleText}\small\ttfamily\color{bgLightGray}}
{\end{raggedright}\end{shaded}\par%
%\ifx\tmp\@nnil{\relax}\else{\vspace{-0.25cm}\captionof{consoleText}{\consoletextCaption}\vspace{0.25cm}\label{\consoletextLabel}}\fi
\ifthenelse{\equal{\tmp}{}}{}{\vspace{-0.25cm}\captionof{consoleText}{\consoletextCaption}\vspace{0.25cm}\label{\consoletextLabel}}
}


%%% Font %%%
\usepackage{fontspec}
\setsansfont{IBMPlexSans}[                       % set up custom font
    Extension = .otf,
    %Path = style/fonts/,
    UprightFont = *-Light,
    BoldFont = *-SemiBold,
    ItalicFont = *-LightItalic,
    BoldItalicFont = *-SemiBoldItalic
]
\renewcommand{\familydefault}{\sfdefault}	

\setlength{\parskip}{0.4em}
\setlength{\parindent}{0em}
%\setenumerate{font=\bfseries}


%%% Header & Footer %%%
\usepackage{fancyhdr}

\pagestyle{fancy}

\fancyfoot{} 
\fancyhf{}

\renewcommand{\sectionmark}[1]{\markboth{\MakeUppercase{\thesection.\ #1}}{}}
%\fancyhead[RE]{\includegraphics[height=2em]{images/ichikoh.png}}
\fancyhead[RO]{\nouppercase{\texttt{\rightmark}}\sectionmark}   
\renewcommand{\headrulewidth}{0pt}

\fancyfoot[LE,RO]{\texttt{\thepage}}
%\fancyfoot[RE]{\texttt{Ichikoh Industries Ltd.}}
%\fancyfoot[LO]{\texttt{Python Stress Documentation v0.9 }}   


\fancypagestyle{plain}{
\fancyhf{}
  \fancyhead{}
  \fancyfoot{}
}

%%% Remark %%%

\makeatletter
\newenvironment{beware}[1][\@nil]
{	\def\tmp{#1}%
    \setlength{\OuterFrameSep}{0pt}						% no space around the text
	\setlength{\FrameSep}{1mm}							% just a bit of colored space around the text
	\definecolor{shadecolor}{rgb}{1.00,0.80,0.80}		% background color for remarks
	\begin{leftbar}\noindent{}%                         % test for option or not
	\ifx\tmp\@nnil{}\else{\textbf{#1 : }}\fi}           % taken from https://tex.stackexchange.com/questions/217757/special-behavior-if-optional-argument-is-not-passed
{\end{leftbar}\par}
\makeatother

%%% Console %%%

\definecolor{fgDarkRed}{RGB}{91,27,22}          % text color in console (draft mode)
\definecolor{fgDarkerRed}{RGB}{51,8,6}          % background color in console
\definecolor{fgVeryLightRed}{RGB}{248,226,224} 

\RequirePackage{pgf}
\pgfkeys{
 /consoletext/.is family, /consoletext,
 caption/.estore in = \consoletextCaption,
 label/.estore in = \consoletextLabel,
}



\begin{document}
\maketitle
%\begin{beware}[\textcolor{red}{Important Remark}]
%\end{beware}

    \section{Certaines langues sont elles plus difficiles à analyser que d’autres ?}
L'objectif de ce projet est d'analyser et de comprendre les raisons pour lesquelles des analyseurs syntaxiques partageant la même architecture et entraînés sur la même quantité de données obtiennent des performances très différentes sur différentes langues. On peut observer ce phénomène dans la Table 2 qui présente les performances calculées à l'aide des mesures LAS (Labeled Accuracy Score) et UAS (Unlabeled Accuracy Score) obtenues par un analyseur sur 36 langues différentes.

\begin{table}[!h]
        \centering
        \begin{tabular}{ccc|ccc|ccc}
            \toprule
            \textbf{lang} & \textbf{las} & \textbf{uas} & \textbf{lang} & \textbf{las} & \textbf{uas} & \textbf{lang} & \textbf{las} & \textbf{uas} \\
            \midrule
            da & 66.18 & 73.06 & zh & 46.76 & 55.60 & pl & 68.76 & 80.69 \\
            hr & 58.88 & 69.67 & lv & 51.46 & 59.32 & sv & 64.52 & 71.11 \\
            id & 69.99 & 75.63 & he & 64.82 & 70.06 & cs & 69.77 & 77.40 \\
            ar & 60.88 & 69.78 & ko & 46.06 & 55.00 & nl & 54.92 & 63.69 \\
            eu & 47.94 & 58.17 & ja & 71.54 & 82.71 & hu & 57.21 & 69.24 \\
            it & 74.82 & 81.01 & ca & 64.79 & 71.13 & bg & 68.08 & 78.14 \\
            fa & 47.70 & 56.17 & en & 66.57 & 71.22 & vi & 48.40 & 49.62 \\
            es & 66.09 & 71.01 & pt & 70.84 & 74.55 & sl & 49.46 & 61.51 \\
            ro & 57.23 & 68.61 & et & 59.04 & 73.09 & nno & 73.58 & 78.30 \\
            de & 58.12 & 64.19 & fr & 68.40 & 73.46 & nob & 66.22 & 73.93 \\
            hi & 64.03 & 72.54 & el & 69.96 & 76.58 &  &  &  \\
            \bottomrule
        \end{tabular}
        \caption{LAS/UAS calculées pour les différentes langues (sans les \textit{configurational features})}
        \label{tab:lasuas}
    \end{table}


    \subsection{Variables explicatives}

    \subsubsection{Observations effectuées sur le corpus d'apprentissage}
    \subsubsection{Complexité selon M. Parkvall}

    Dans son papier The simplicity of creoles in a cross-linguistic perspective sorti en 2008, Mikael Parkvall s’intéresse à quantifier la complexité des langues. Il part du postulat qu’une expression est d’autant plus complexe qu’elle implique de règles, c’est-à-dire qu’elle requiert une longue description. Ainsi, l'hypothèse de base de l’auteur est la suivante : une langue complexe est une langue avec des constructions plus complexes. Il explore un aspect de complexité structurelle.

    Prenons par exemple la voix passive. Lorsqu’elle existe dans une langue, il faut pouvoir définir comment passer de la voie active à la voix passive, ce qui exige une explication de règle supplémentaire. Une langue qui possède une voix passive est donc, en ce qui concerne cette construction spécifique, plus complexe qu’une autre n’en possédant pas. Si on énumère donc un grand nombre de \og constructions complexes \fg{}, la langue la plus complexe sera celle qui en compte le plus grand nombre.

    Pour extraire les \og constructions complexes \fg{} qu’on peut trouver dans une langue, Parkvall utilise le set de données World Atlas of Linguistic Structures (WALS) publié en 2005 par Haspelmath et al.. Il choisit 155 langues parmi plus de 2 500, et 47 caractéristiques parmi plus de 140.

    \paragraph{Choix des caractéristiques}

    Il exclut des caractéristiques selon un raisonnement défendu dans son papier et qui s’efforce de mettre la majorité des linguistes d’accord sur le fait qu’une caractéristique apporte de la complexité. Il retient les caractéristiques suivantes : \par

    \begin{table}[!h]
        \centering
        \rowcolors{3}{bgLightGray}{}
        \begin{tabulary}{\textwidth}{LLL}
            \toprule
            \multicolumn{3}{c}{\textbf{Caractéristiques du WALS}} \\
            \midrule
            Size of consonant inventories & Distance contrast in demonstratives & Morphological imperative \\
            Size of vowel quality inventories & Gender in pronouns & Morphological optative \\
            Phonemic vowel nasalization & Politeness in pronouns & Grammaticalized evidentiality distinctions \\
            Complexity of syllable structure & Person marking on adpositions & Both indirect and direct evidentials \\
            Tone & Comitative ≠ instrumental & Non-neutral marking of full NPs \\
            Overt marking of direct object & Ordinals exist as a separate class beyond ‘first’ & Non-neutral marking of pronouns \\
            Double marking of direct object & Suppletive ordinals beyond ‘first’ & Subject marking as both free word and agreement \\
            Possession by double marking & Obligatory numeral classifiers & Passive \\
            Overt possession marking & Possessive classification & Antipassive \\
            Reduplication & Conjunction ‘and’ ≠ adposition ‘with’ & Applicative \\
            Gender & Difference between nominal and verbal conjunction & Obligatorily double negation \\
            Number of genders & Grammaticalized perfective/imperfective & Asymetric negation \\
            Non-semantic gender assignment & Grammaticalized past/non-past & Equative copula ≠ Locative copula \\
            Grammaticalized nominal plural & Remoteness distinctions of past & Obligatorily overt equative copula \\
            Definite articles Indefinite articles & Morphological future &  \\
            Inclusivity (in either pronouns or verb morphology) & Grammaticalized perfect &  \\
            \bottomrule
        \end{tabulary}
        \caption{Liste des caractéristiques extraite directement du WALS}
        \label{tab:0}
    \end{table}

    Il ajoute à ces caractéristiques, d’autres données “résiduelles” d’auteurs contributeurs au WALS que sont les suivantes : \par

   \begin{table}[!h]
        \centering
        \rowcolors{3}{bgLightGray}{}
        \begin{tabulary}{\textwidth}{LLL}
            \toprule
            \multicolumn{3}{c}{\textbf{Caractéristiques d'auteurs du WALS}} \\
            \midrule
            Demonstratives marked for number & Demonstratives marked for gender & Demonstratives marked for case \\
            Total amount of verbal suppletion & Alienability distinctions &  \\
            \bottomrule
        \end{tabulary}
        \caption{Liste de caractéristiques proposées par les auteurs du WALS}
        \label{tab:1}
    \end{table}

    Ainsi qu’une donnée de Harley and Ritter (2002) à laquelle il a eu accès : \par

\begin{table}[!h]
        \centering
        \begin{tabulary}{\textwidth}{LLL}
            \toprule
            \multicolumn{3}{c}{\textbf{Caractéristique de Harley et Ritter}} \\
            \midrule
             & Number of pronominal numbers &  \\
            \bottomrule
        \end{tabulary}
        \caption{Une caractéristique accessible, inspirée de Harley et al. (2002)}
        \label{tab:2}
    \end{table}

    Les valeurs de ces caractéristiques ont toutes été traduite par l’auteur comme des valeurs entre 0 et 1 : \par
    "Oui ou non" devient 0 ou 1 (avec parfois des 0.5). \par
    Des valeurs d'intensité comprises entre 1 et 4 sont compréssée en : 0, 0.25, 0.5, 0.75 et 1. \par
    Des valeurs catégoriques comme la classification en “simple”, “modérément complexe” et “complexe” sont traduites en 0, 0.5, 1.

    \paragraph{Choix des langues}

    L’auteur s’attèle à choisir des langues dont les annotations ne sont pas lacunaires pour ces caractéristiques. Pour une langue i donnée :
    \begin{equation}
    \text { Score }_{i}=\frac{\sum_{k=1}^{L} \text { contribution }_{k}}{L}
    \end{equation}
    k représente une caractéristique. Chaque langue compte un nombre différent L de caractéristiques (parmi celles choisies) effectivement annontées pour cette langue. \par
    N.B.: en effet, de même que pour les caractéristiques que nous extrayons directement de nos données, Parkvall note que le set de données WALS n’est pas identiquement distribué : les langues ne sont pas identiquement annotées pour les caractéristiques proposées…

\end{document}